\documentclass[10pt,a4paper,draft,titlepage,onecolumn]{book}
\usepackage[left=0.5in, top=1in, bottom=1in, right=0. 5in]{geometry}
\usepackage[utf8]{inputenc}
\usepackage{amsmath}
\usepackage{amsfonts}
\usepackage{amssymb}
\usepackage{soul}

\usepackage[english]{babel}
\usepackage{index}
\makeindex
\usepackage{url}
\newtheorem{theorem}{Theorem}[section]
\newtheorem{definition}{Definition}[subsection]
\newtheorem{remark}{Remark}
\newtheorem{example}{Example}[section]
\newtheorem{property}{Property}[section]
\newtheorem{exercise}{Exercise}[section]
\newtheorem{lemma}[theorem]{Lemma}


\author{Arvid Claassen}
\date{March 2023}
\title{Math for Computer Scientists}
\begin{document}
\maketitle
\tableofcontents
\chapter*{Preface}
Welcome to the chapter to no one reads. So let me keep it short.
What follows is my attempt to write a book to give you insight in several math topics. \\
The latest version of this book is available at \url{https://github.com/nibblus/discrete_math}\\ And will grow over time. 
It is free, but feel free to buy me a coffee.  \url{https://www.buymeacoffee.com/arvidclaassen}
Also note theat English is not my mother tongue, so if you find any Yoda-like gibberish please let me know at \url{discrete_math@12test.be}




\chapter{Fundamentals of Logic}
\section*{Introduction}
Logic is the core all mathematical reasoning. The rules of logic specify the meaning of mathematical statements and how transform one into another. 
In this chapter we'll formalize reasoning. 
We'll start with propositions, the most basic building blocks in logic. And build up to a systematic model for reasoning.
As logic and computer science go hand in hand, a \textit{lot} of section will refer on how to apply them the computer languague or digital techniques.

\section{Core Operations and Properties}
\begin{definition}[Proposition]
A declarative sentence that is either True or False, but not both. Other sources may use the term \textit{statement} instead. Both are interchangeable. 
\end{definition}
Propositions are notated as lowercase letters:
\begin{itemize}
 \item[$p$:]Beethoven wrote 9 symphonies. (True)
 \item[$q$:]Einstein lived in Germany and the United States. (True)
 \item[$r$:]$ 1 + 3 = 5 $ (False)
\end{itemize} 
In order to confirm that propositions $p$ and $q$ are True, you probably have to check an historical source. For now we ask you to trust us. 
Proposition $r$ is clearly wrong, however it is a valid proposition with truth value False. \\ \\
Not every sentence is a proposition. None of the following sentences is a proposition:
\begin{itemize}
 \item[$s$:]Get out! (No truth value)
 \item[$t$:]$2x+1=5$ (Might be True or False, depending on the value of x)
 \item[$u$:]Such a nice flower. (No truth value)
 \item[$v$:]Every even positive integer greater than 2 is the sum of two prime numbers. \textit{(Goldbach’s Conjecture)} (Unknown)
\end{itemize}
At the time of writing \textit{Goldbach's Conjecture} is still unproven, therefore its truth value is Unknown. If at the time of reading someone has (dis)proven the conjecture, then $v$ will be a proposition. For the first and only time in this book: \textit{the proof is left to the reader.}

\begin{definition}[Primitive Proposition]
A proposition that can not be broken down into simpler propositions. Other sources may use the term \textit{atomic} instead. Both are interchangeable. 
\end{definition}

\begin{definition}[Compound Proposition]
A proposition that contains two or more primitive propositions.  
\end{definition}
$p$ and $r$ are primitive propositions. 
$q$ is a compound proposition and can be broken down into $q_1$: \textit{Einstein lived in Germany} and $q_2$:\textit{ Einstein lived in the United States}. In order to determine the truth value of $q$ you need to know the truth value of both $q_1$ and $q_2$  \\


\begin{definition}[Truth Table]
A breakdown of all the possible truth values of a proposition. 
\end{definition}
The most basic truth table is that of a primitive proposition.
\begin{center}
\begin{tabular}{ |c| }
 \hline
 $p$ \\
 \hline
 False \\ 
 True  \\
 \hline
\end{tabular}
\end{center}
For reasons of brevity we replace the values False and True by 0 and 1 respectively. The above table can be rewritten as: 
\begin{center}
\begin{tabular}{ |c| }
 \hline
 $p$ \\
 \hline
 0 \\ 
 1 \\
 \hline
\end{tabular}
\end{center}


\section{Unary Logical Operators}
\begin{definition}[Unary Logical Operator]
An operation that works with only one proposition.
\end{definition}

\begin{definition}[Negation] Negation is a unary operation that turns a proposition $p$ into to another proposition "not $p$", written {$\neg$}$p$. Other textbook may use the term complement: not $p$ is the complement of $p$. Both are interchangeable. 
\end{definition} 
We say  \textit{{$\neg$}$p$ is the negation of $p$}. \\
Intuitively, if $p$ is True, then $\neg${$p$} is False, and conversely, if $\neg${$p$} is True, then $p$ is False.

The truth table of negation is as follows:
\begin{center}
\begin{tabular}{ |c| c| }
 \hline
 $p$ & {$\neg$}$p$ \\
 \hline
 0 & 1 \\ 
 1 & 0 \\
 \hline
\end{tabular}
\end{center}
For example, proposition $p$ is \textit{4 is an even number}. We can easily determine that $p$ is True.\\
$\neg${$p$} is  \textit{not(4 is an even number)}, which can be rephrased as \textit{4 is not an even number} or as \textit{4 is an odd number}.\\
Independent of the phrasing: $\neg${$p$} is False 

\begin{property}[Double negation] If p is True, then {$\neg$}{$\neg$}p is True. If p is False, then {$\neg$}$\neg$p is False
\end{property}
We can prove this property with this truth table:
\begin{center}
\begin{tabular}{ |c| c|c| }
 \hline
 $p$ & {$\neg$}$p$ & {$\neg$}{$\neg$}$p$  \\
 \hline
 0 & 1 & 0 \\ 
 1 & 0 & 1\\
 \hline
\end{tabular}
\end{center}
In the Sitcom Big Bang Theory episode S5E10 main character Sheldon Cooper says: \textit{"I would not object to us no longer characterizing you as not my girlfriend."} Linguistically, \textit{to object }is the opposite of \textit{to accept}, so Sheldon's words translate to: \textit{"I do not not accept us no longer characterizing you as not my girlfriend."} 
Which is the same as: \textit{"I do accept us no longer characterizing you as not my girlfriend."} 
Which is the same as: \textit{"I do accept us characterizing you as my girlfriend."}\\\\
Logically speaking, Pink Floyd's song title \textit{"We don't need no education"} means \textit{"We need education"}. Other song titles that suffer from the same discrepancy between logic and spoken language include \textit{"It’s Not Unusual" }by Tom Jones, \textit{"(I Can’t Get No) Satisfaction" }by the Rolling Stones
, and \textit{"Ain’t No Mountain High Enough" }by Marvin Gaye and Tammi Terrell.


\begin{definition}[Contradiction] $\bot p$ is always False.
\end{definition}  
\begin{center}
\begin{tabular}{ |c|c| }
 \hline
 $p$ &  ${\bot}p $ \\
 \hline
 0 &  0 \\
 1 &  0\\
 \hline
\end{tabular}
\end{center}
By itself this truth function makes little sense. Why would your ever reason with something that is always false?
\textit{Proof by contradiction } is  is a form of proof that establishes the truth or the validity of a proposition, by showing that assuming the proposition to be false leads to a contradiction. 
A mathematical proof employing proof by contradiction usually proceeds as follows:
\begin{itemize}
\item The proposition to be proved is $p$.
\item We assume $p$ to be false, i.e., we assume $\neg p$.
\item It is then shown that $\neg p$implies falsehood. 
\item Since assuming $\neg p$ to be false leads to a contradiction, it is concluded that $ p$is in fact true.
\end{itemize}
A famous proof by contradiction is: $\sqrt{2}$ is an irrational number, i.e. it can not be written as  $\frac{p}{q}$ where $p$ and $q$ are integers.
\begin{itemize}
\item Let's assume $\sqrt{2}$  to  be rational, then $\sqrt{2} = \frac{p}{q}$, where $p$ and $q$ are integers without common factors and $q \neq 0$
\item So, $ 2 = \frac{p^2}{q^2} $   
\item So, $ p^2 = 2{q^2} $ which means $p$ is even and can be rewritten as $ 2k$ ($k$ is an integer)
\item So,  $ p^2 = 4k^2$, and also $ q^2 = 2k^2$
\item So, $q$ is even
\item Contradiction: Both $p$ and $q$ are even, so they share factor 2, which contradicts that they do not have common factors.  
\item So  \textit{${\neg}(\sqrt{2}$ is an irrational number)} is False, thus  \textit{$\sqrt{2}$ is an irrational number }is True; 
\end{itemize}
In programming you might accidentally end up with a contradiction . More often than not they indicate a programming mistake. \\ \\
\textbf{IF x $<$ 3 AND  x $>$5  THEN \textit{dosomething()}}\\
which is the same as \\
\textbf{IF False THEN \textit{dosomething()}}\\
which equals an empty statement \textit{dosomething()} will never be executed.
\begin{definition}[Tautology]
$\top p $ is always True.
\end{definition}
\begin{center}
\begin{tabular}{ |c|c| }
 \hline
 $p$ & $\top p$  \\
 \hline
 0 & 1 \\
 1 & 1\\
 \hline
\end{tabular} 
\end{center} 
An example \textit{the ball is red, or the ball is not red} is always True, regardless of the colour of the ball.
In linguistics a \textit {tautology} means \textit{useless repetion}, also {saying the same twice}. In the sentence \textit{This explanation should be adequate enough}, the word \textit{enough} is only a repetition of \textit{adequate}. Removing it from the sentence does not change the meaning of the sentence. \\
In programming you might accidentally encounter tautologies. As with contradictions they often indicate a programming mistake. \\
\textbf{IF x $<$= 0 OR  x $>$=0 THEN \textit{dosomething()}}\\
which is the same as \\
\textbf{IF TRUE THEN \textit{dosomething()}}\\
which is the same as \\
\textbf{dosomething()}\\
thus omitting the condition altogether. \textbf{dosomething()} will always be executed.

\begin{center} 
\textbf{The tautology Limerick}.\\
There once was a fellow in Perth \\
Who was born on the day of his birth. \\
He was married, they say, \\
On his wife's wedding day, \\
And he died when he quitted this earth 
\end{center}

\begin{definition}[Identity]
$=p$ is equal to $p$
\end{definition}

\begin{center}
\begin{tabular}{ |c|c| }
 \hline
 $p$ & $=p$  \\
 \hline
 0 &0 \\
 1 & 1\\
 \hline
\end{tabular} 
\end{center} 
Identity is only mentioned for reasons of completeness. This operator does not change the truth value, and can be omitted at all times.

\section{Binary Logical Operators}
\subsection*{Introduction}
\begin{definition}[Binary Logical Operator]
An operation that works with two propositions. Other textbooks may use the terms \textit{Truth Function}, \textit{Boole Function}.
\end{definition}
Truth tables for binary logical operators have an extra column for the truth values of $q$ and two extra rows to include every combination of truth values of $p$ and $q$. \\
Imagine $\boxplus$ is a binary logical operator, then the truth table would be:\\
\begin{tabular}{ |c|c|c| }
 \hline
 $p$ & $q$ & $p \boxplus q$ \\
 \hline 
 0 & 0 & ? \\
 1 & 0 & ? \\
 0 & 1 & ? \\
 1 & 1 & ? \\
 \hline
\end{tabular} \\
For every of the 16 possible outcomes ranging from (0,0,0,0) to  (1,1,1,1), there is an operator. However, as we'll see further on: not all operators are being used as often.
\begin{definition}[Converse proposition] Result of reversing the two constituent propositions.
\end{definition}
The converse of  $p{\boxplus}q$  is  $q{\boxplus}p$. \\ \\ 
\textbf{Important:} To simplify the definition in the next section, we'll determine when the operator results in truth value True. It is False in all other cases.
\subsection{Operators}
\begin{definition}[Conjunction, And Operator] $p{\wedge}q$, read ``p and q'', is True when both $p$ and $q$ are True.
\end{definition} 
\begin{tabular}{ |c|c|c| }
 \hline
 $p$ & $q$ &  $p{\wedge}q$  \\
 \hline
 0 & 0 & 0 \\
 1 & 0 & 0 \\
 0 & 1 & 0 \\
 1 & 1 & 1 \\
 \hline
\end{tabular}\\
We use the term \textit{conjunction} to clearly distinguish the logical operator \textit{and} and the linguistic connective \textit{and}. Note that the propositions itself do not have to be related: It might no be the best pick up line, but the sentence \textit{``4 is a number and a lion is a mammal''} is a meaningful True compound proposition. \\
\begin{definition}[Inclusive Disjunction, Or Operator]  $p{\vee}q$, is read \textit{``p or q"} is True when $p$ is True, $q$ is True, or both are True. \\\end{definition} 
\begin{tabular}{ |c|c|c| }
 \hline
 $p$ & $q$ &  $p{\vee}q$  \\
 \hline
 0 & 0 & 0 \\
 1 & 0 & 1\\
 0 & 1 & 1\\ 
 1 & 1 & 1\\
 \hline
\end{tabular}\\
The pronoun \textit{Inclusive} denotes that picking both options is allowed.  In spoken language \textit{or} often means \textit{pick one but not both} which is the \textit{Exclusive disjunction} (see below).\\
If you would be make it to the final of the game show \textit{Let's make a deal}, and the hosts presents your with the choice: \textit{Which door would you like to open to reveal your price? Door number  or door number 23?}. The best logical answer would be \textit{Yes} to get the prices behind both doors.

\begin{definition}[Exclusive Disjunction, Exclusive Or, Xor Operation] $p{\oplus}q$ , is read \textit{``p xor q"} is True when either $p$ is True or $q$ is True, but not both. \end{definition}
\begin{tabular}{ |c|c|c| }
 \hline
 $p$ & $q$ &  $p{\oplus}q$  \\
 \hline
 0 & 0 & 0 \\
 1 & 0 & 1\\
 0 & 1 & 1\\
 1 & 1 & 0\\
 \hline
\end{tabular}\\\\
Another way to interpret $p{\oplus}q$ is: \textit{Does not have the same truth value as}.
The pronoun \textit{Exclusive} denotes that picking both options is not allowed. In this case, when having to choose between door number 1 xor 2, you're on your own.

\begin{definition}[Material Implication] $p{\rightarrow}q$, read \textit{``p implies q"} is False when $p$ is True and $q$ is False.  
\end{definition} 
\begin{tabular}{ |c|c|c| }
 \hline
 $p$ & $q$ &  $p{\rightarrow}q$  \\
 \hline
 0 & 0 & 1 \\
 1 & 0 & 0 \\
 0 & 1 & 1 \\
 1 & 1 & 1 \\
 \hline
\end{tabular} \\
Implication is a powerful tool in logical reasoning. If we know that $p{\rightarrow}q$ and $q{\rightarrow}r$, then we can conclude that $p{\rightarrow}r$. This will be proven later on. This kind of reasoning in \textit{one way}. The truth of $p$ implies the truth of $q$. However, if $q$ is True, then we can \textit{not} logically deduce truth value of $p$., e.g. $p$: It rains; $q$ The street is wet.  $p{\rightarrow}q$ now reads: \textit{If it rains then the street is wet}.  The street being wet does not imply that it rains, as there might be other reasons.

\begin{definition}[Converse Implication] $p{\leftarrow}q$, read \textit{``q implies p"} is False when $p$ is False and $q$ is True.  
\end{definition}
\begin{tabular}{ |c|c|c| }
 \hline
 $p$ & $q$ &  $p{\leftarrow}q$  \\
 \hline
 0 & 0 & 1 \\
 1 & 0 & 1\\
 0 & 1 & 0\\
 1 & 1 & 1\\
 \hline
\end{tabular}\\
As defined $p{\leftarrow}q$ is the converse of $q{\rightarrow}p$.
There are situation where $p{\leftarrow}q$ is clearer than $q{\rightarrow}p$, e.g. logical programming.


\begin{definition}[Biconditional] $p{\leftrightarrow}q$, read ``$p$ if and only if  $q$", is True if $p$ and $q$ are both False or both True.
\end{definition}
\begin{tabular}{ |c|c|c| }
 \hline
 $p$ & $q$ &   $p{\leftrightarrow}q$\\
 \hline
 0 & 0 & 1 \\
 1 & 0 & 0\\
 0 & 1 & 0\\
 1 & 1 & 1\\
 \hline
\end{tabular}\\\\
Biconditional is a powerful tool in logical reasoning. If we know that $p{\leftrightarrow}q$ and $q{\leftrightarrow}r$, then we can conclude that $p{\leftrightarrow}r$. This will be proven later on. This kind of reasoning in \textit{two way}. The truth of $p$ implies the truth of $q$ and if $q$ is True, then we can logically deduce truth value of $p$., This operator can be used to simplify compound propositions. \\
Later in this chapter we'll \textit{reason} that:
$(p{\vee}q)\wedge{\neg}(p{\wedge}q)\wedge({\neg}p{\vee}{\neg}q) \leftrightarrow p \oplus q$\\
This means that whenever you come across a compound proposition $(p{\vee}q)\wedge{\neg}(p{\wedge}q)\wedge({\neg}p{\vee}{\neg}q)$, you are allowed to replace it by $p \oplus q$, and the truth value of the proposition will remain identical.


\begin{definition}[Alternative Denial, Nand, Sheffer Stroke] $p{\uparrow}q$, read  ``$p$ \textit{nand} $q$"  is True if not both are True.
\end{definition}
\begin{tabular}{ |c|c|c| }
 \hline
 $p$ & $q$ &  $p{\uparrow}q$  \\
 \hline
 0 & 0 & 1 \\
 1 & 0 & 1\\
 0 & 1 & 1\\
 1 & 1 & 0\\
 \hline
\end{tabular}  \\\\
The word \textit{nand} does not exist in the English language, it stands for \textit{not and}. which make the liguisitic interpretation: When provided a choice between two options: you are allowed to choose none or one, but not both.\\
The operator is named after \textbf{Henry Sheffer} who proved that Boolean algebra could be defined using a only Alternative Denial. This proof can not be overrated as it forms the basis of chips building techniques.


\begin{definition}[Material Non Implication]$p{\nrightarrow}q$, read ``p but not q'', is True when p is True and q is False.
\end{definition}
\begin{tabular}{ |c|c|c| }
 \hline
 $p$ & $q$ & $p{\nrightarrow}q$  \\
 \hline
 0 & 0 & 0 \\
 1 & 0 & 1\\
 0 & 1 & 0\\
 1 & 1 & 0\\
 \hline
\end{tabular} \\\\
$p{\nrightarrow}q$ is True as long as $p$ does not imply $q$, hence \textit{Non Implication}


\begin{definition}[Converse Non Implication]$p{\nleftarrow}q$, read ``$q$ does not imply $p$'', is true when q is True and p is False.
\end{definition}  
\begin{tabular}{ |c|c|c| }
 \hline
   $p$ & $q$ &  $p{\nleftarrow}q$  \\
 \hline
 0 & 0 & 0 \\
 1 & 0 & 0\\
 0 & 1 & 1\\
 1 & 1 & 0\\
 \hline
\end{tabular}\\\\
As with the converse implication, there are situations where $q{ \nleftarrow}p$  is more convenient than $p{\nrightarrow}q$.



\begin{definition}[Joint Denial, Nor]$p{\downarrow}q$, read `` $p$ nor $q$'', is True when $p$ and $q$ are both False.
\end{definition}
\begin{tabular}{ |c|c|c| }
 \hline
   $p$ & $q$ &  $p{\downarrow}q$ \\
 \hline
 0 & 0 & 1 \\
 1 & 0 & 0\\
 0 & 1 & 0\\
 1 & 1 & 0\\
 \hline
\end{tabular}\\\\
Nor can also be read ``\textit{neither $p$ or $q$}''; It is True when both can be denied, hence \textit{joint denial}.\\

For reasons of completeness we mention the remaining binary logical operators.
Although these six operators are binary in the strict sense of the word, their underlying function is unary as they ignore at least one operand. 


\begin{definition}[Proposition of p] p, read ``p'', is True when p its True.
\end{definition}
\begin{tabular}{ |c|c|c| }
 \hline
 $p$ & $q$ &  $p$ \\
 \hline
 0 & 0 & 0 \\ 
 1 & 0 & 1\\
 0 & 1 & 0 \\
 1 & 1 & 1\\
 \hline
\end{tabular}\\\\


\begin{definition}[Proposition of q] q, read ``q'', is True when q its True.
\end{definition}
\begin{tabular}{ |c|c|c| }
 \hline
 $p$ & $q$ &  $q$ \\
 \hline
 0 & 0 & 0 \\ 
 1 & 0 & 0\\
 0 & 1 & 1 \\
 1 & 1 & 1\\
 \hline
\end{tabular}\\\\


\begin{definition}[Negation of p] ${\neg}p$, read ``not $p$'', is True when $p$ is False.
\end{definition}
\begin{tabular}{ |c|c|c| }
 \hline
 $p$ & $q$ &  ${\neg}p$  \\
 \hline
 0 & 0 & 1 \\
 1 & 0 & 0\\
 0 & 1 & 1\\
 1 & 1 & 0\\
 \hline
\end{tabular}\\\\

\begin{definition}[Negation of q] ${\neg}q$, read ``not $q$'', is True when $q$ is False.
\end{definition}
\begin{tabular}{ |c|c|c| }
 \hline
 $p$ & $q$ &  ${\neg}q$  \\
 \hline
 0 & 0 & 1 \\
 1 & 0 & 0\\
 0 & 1 & 1\\
 1 & 1 & 0\\
 \hline
\end{tabular}\\\\

\begin{definition}[Contradiction] $p{\bot}q$, ``read \textit{contradiction}'', is always False.
\end{definition}
\begin{tabular}{ |c|c|c| }
 \hline
 $p$ & $q$ &  $p{\bot}q$  \\
 \hline
 0 & 0 & 0 \\
 1 & 0 & 0\\
 0 & 1 & 0\\
 1 & 1 & 0\\
 \hline
\end{tabular}\\\\

\begin{definition}[Tautology] $p{\top}q$, read ``\textit{tautology}'', is always True.
\end{definition}
\begin{tabular}{ |c|c|c| }
 \hline
 $p$ & $q$ &  $p{\top}q$  \\
 \hline
 0 & 0 & 1\\
 1 & 0 & 1\\
 0 & 1 & 1\\
 1 & 1 & 1\\
 \hline
\end{tabular}\\\\



\section{Proof using Truth Tables}
Truth tables can be used to determine the truth value of any compound proposition, as follows:

\begin{itemize}
 \item Add a column for every basic proposition 
 \item Add a column for every unary and binary operator.
 \item Add a final column for the compound proposition 
 \item Add a row with every possible value combination of the basic propositions, i.e. $2^n$ rows where $n$ is the number of different atomic propositions. 
\end{itemize}


To determine the truth value of the compound statement {$\neg$}$p${$\wedge$}$q${$\wedge$}{$\neg$}$r$, we create a truth table as follows:
There are 3 atomic propositions: $p$, $q$, $r$. Add a column for each of them.\\

\begin{center}
\begin{tabular}{ |c|c|c| }
 \hline
 p & q & r \\
 \hline
\end{tabular}
\end{center}
Add a column for every unary operator, {$\neg$}p, {$\neg$}r.\\
\begin{center}
\begin{tabular}{ |c|c|c|c|c|}
 \hline
 {$\neg$} & $p$ &$q$& {$\neg$}&$r$ \\
 \hline
\end{tabular}
\end{center}
Add a column for every unary and binary operator.
\begin{center}
\begin{tabular}{ |c|c|c|c|c|c|c|}
 \hline
 {$\neg$} & $p$ &{$\wedge$}&$q$&{$\wedge$} & {$\neg$}&$r$ \\
 \hline
\end{tabular}
\end{center}
Add a column for every unary and binary operator.
\begin{center}
\begin{tabular}{ |c|c|c|c|c|c|c|}
 \hline
 {$\neg$} & $p$ &{$\wedge$}&$q$&{$\wedge$} & {$\neg$}&$r$ \\
 \hline
\end{tabular}
\end{center}
Add a row with every possible value combination of the basic proposition. \\
For $p$: 0, 0, 0, 0, 1, 1, 1, 1\\
For $q$: 0, 0, 1, 1, 0, 0, 1, 1\\
For $r$: 0, 1, 0, 1, 0, 1, 0, 1\\
\begin{center}
\begin{tabular}{ |c|c|c|c|c|c|c| }
\hline
 {$\neg$} & $p$ & {$\wedge$} & $q$ & {$\wedge$} & {$\neg$}  & $r$ \\
\hline
          & 0   &            & 0   &            &           &0\\
          & 0   &            & 0   &            &           &1\\
          & 0   &            & 1   &            &           &0\\
          & 0   &            & 1   &            &           &1\\
          & 1   &            & 0   &            &           &0\\
          & 1   &            & 0   &            &           &1\\
          & 1   &            & 1   &            &           &0\\
          & 1   &            & 1   &            &           &1\\
 \hline
\end{tabular}
\end{center}

Fill in the truth value of the unary operators.
\begin{center}
\begin{tabular}{ |c|c|c|c|c|c|c| }
\hline
 {$\neg$} & $p$ & {$\wedge$} & $q$ & {$\wedge$} & {$\neg$}  & $r$ \\
\hline
       \textbf{1}   & \st{0}   &            & 0   &            &       \textbf{1}    &\st{0}\\
       \textbf{1}   & \st{0}   &            & 0   &            &       \textbf{0}    &\st{1}\\
       \textbf{1}   & \st{0}   &            & 1   &            &       \textbf{1}    &\st{0}\\
       \textbf{1}   & \st{0}   &            & 1   &            &       \textbf{0}    &\st{1}\\
       \textbf{0}   & \st{1}   &            & 0   &            &       \textbf{1}    &\st{0}\\
       \textbf{0}   & \st{1}   &            & 0   &            &       \textbf{0}    &\st{1}\\
       \textbf{0}   & \st{1}   &            & 1   &            &       \textbf{1}    &\st{0}\\
       \textbf{0}   & \st{1}   &            & 1   &            &       \textbf{0}    &\st{1}\\
 \hline
\end{tabular}
\end{center}

Fill in the truth values of the binary operators.\\ First, we'll fill in  {$\neg$} $p$ {$\wedge$}  $q$


\begin{center}
\begin{tabular}{ |c|c|c|c|c|c|c| }
\hline
 {$\neg$} & $p$ & {$\wedge$} & $q$ & {$\wedge$} & {$\neg$}  & $r$ \\
\hline
       \st{1}   & \st{0}   &  \textbf{0}          & \st{0}   &            &       \textbf{1}    &\st{0}\\
       \st{1}   & \st{0}   &  \textbf{0}          & \st{0}   &            &       \textbf{0}    &\st{1}\\
       \st{1}   & \st{0}   &  \textbf{1}          & \st{1}   &            &       \textbf{1}    &\st{0}\\
       \st{1}   & \st{0}   &  \textbf{1}          & \st{1}   &            &       \textbf{0}    &\st{1}\\
       \st{0}   & \st{1}   &  \textbf{0}          & \st{0}   &            &       \textbf{1}    &\st{0}\\
       \st{0}   & \st{1}   &  \textbf{0}          & \st{0}   &            &       \textbf{0}    &\st{1}\\
       \st{0}   & \st{1}   &  \textbf{0}          & \st{1}   &            &       \textbf{1}    &\st{0}\\
       \st{0}   & \st{1}   &  \textbf{0}          & \st{1}   &            &       \textbf{0}    &\st{1}\\
 \hline
\end{tabular}
\end{center}

Secondly we'll fill in the final column, which is the result of this truth table
{$\neg$} $p$  {$\wedge$}  $q$  {$\wedge$}  {$\neg$}   $r$ \\

\begin{center}
\begin{tabular}{ |c|c|c|c|c|c|c| }
\hline
 {$\neg$} & $p$ & {$\wedge$} & $q$ & {$\wedge$} & {$\neg$}  & $r$ \\
\hline
       \st{1}   & \st{0}   &  \st{0}          & \textbf{0}   &            &       \textbf{1}    &\st{0}\\
       \st{1}   & \st{0}   &  \st{0}          & 0   &            &       \textbf{0}    &\st{1}\\
       \st{1}   & \st{0}   &  \st{1}          & 1   &            &       \textbf{1}    &\st{0}\\
       \st{1}   & \st{0}   &  \st{1}          & 1   &            &       \textbf{0}    &\st{1}\\
       \st{0}   & \st{1}   &  \st{0}          & 0   &            &       \textbf{1}    &\st{0}\\
       \st{0}   & \st{1}   &  \st{0}          & 0   &            &       \textbf{0}    &\st{1}\\
       \st{0}   & \st{1}   &  \st{0}          & 1   &            &       \textbf{1}    &\st{0}\\
       \st{0}   & \st{1}   &  \st{0}          & 1   &            &       \textbf{0}    &\st{1}\\
 \hline
\end{tabular}
\end{center}

The compound statement {$\neg$}$p${$\wedge$}$q${$\wedge$}{$\neg$}$r$ is True when {$\neg$}p, q, {$\neg$}r are True.

\begin{center}
\begin{tabular}{ |c|c|c|c|c|c| }

 \hline
 p & q & r &  {$\neg$}p &{$\neg$}r & {$\neg$}p{$\wedge$}q{$\wedge$}{$\neg$}r \\
 \hline
 0 & 0 & 0 & 1 & 1 & 0\\
 1 & 0 & 0 & 0 & 1 & 0\\
 0 & \textbf{1} & 0 & \textbf{1} & \textbf{1} & \textbf{1}\\
 1 & 1 & 0 & 0 & 1 & 0\\
 0 & 0 & 1 & 1 & 0 & 0\\
 1 & 0 & 1 & 0 & 0 & 0\\
 0 & 1 & 1 & 1 & 0 & 0\\
 1 & 1 & 1 & 0 & 0 & 0\\
 \hline
\end{tabular}
\end{center}


\begin{example}
p $\wedge$ q is True, as both p and q are True.
p $\wedge$ r is False, because not both p and r are True.
\end{example}

\begin{property}[Conjuction is commutative] The truth value of p $\wedge$  q is equal to the truth value of q $\wedge$ p.
\end{property}
In a compound conjuction p $\wedge$ q $\wedge$ r, the order of propositions has no effect on the truth value of overall truth value. The following propositions all have the same truth value:
p $\wedge$ q $\wedge$ r,
p $\wedge$ r $\wedge$ q,
q $\wedge$ p $\wedge$ r,
q $\wedge$ r $\wedge$ p,
r $\wedge$ q $\wedge$ p,
r $\wedge$ p $\wedge$ q
\begin{property}[Idempotency of p $\wedge$  p] If p is True, p $\wedge$ p is True. If p is False, then p $\wedge$ p is False
\end{property}
Note that Conjunction itself is not idempotent: 
Not in all cases is  p $\wedge$ q equal to p, e.g. when p is True and q is False: p $\wedge$ q is False, but p is True
Not in all cases is  p $\wedge$ q equal to q. 
But in all cases   p $\wedge$ p is equal to p 

These two properties of disjunction can be used to simplify compound proposition into simpler ones or even a primitive one. This is in fact the core of logical reasoning. 
If somewhere in logic you end up with a proposition p $\wedge$ q $\wedge$ q $\wedge$ p, then by using these properties you can simplify this.

p $\wedge$ q $\wedge$ q $\wedge$ p
equals to (Idempotency of q $\wedge$ q) 
p $\wedge$ q $\wedge$ p
equals to  (Conjuction is commutative)
p $\wedge$ p $\wedge$ q
equals to (Idempotency of p $\wedge$ p)
p $\wedge$ q

The logical reduction from  p $\wedge$ q $\wedge$ q $\wedge$ p to p $\wedge$ q is valid in all cases, no matter what the truth value p or q.				

p $\wedge$ q $\wedge$ q $\wedge$ p
equals to (Conjuction is commutative)
p $\wedge$ p $\wedge$ q $\wedge$ q
equals to  (Idempotency of p $\wedge$ p)
p $\wedge$ q $\wedge$ q
equals to (Idempotency of q $\wedge$ q)
p $\wedge$ q

The order in which properties are applied does not matter. 

\begin{property}[p $\wedge$  False]  If any proposition in a disjunctive proposition is False, then the whole proposition is False. In other words: p $\wedge$  False is always False.
\end{property}

This property is often used in computer languages to optimize code.

$$a = 5$$
$$b = 3$$
$$IF  a > 10  and b > 2 THEN do-something$$
		
If both $a > 10  and  b > 2$ are then the do-something part will be executed. 
Typically a programming language will evaluate the IF part from left to right. 
In this case $a > 10$ will be evaluated first. Since this is already False the remainder of the evaluation may be skipped and do-something will no be executed.

\begin{property}[p $\wedge$  True]  If a proposition in a compound disjunctive proposition is True, it can be removed from the proposition. In other words: p $\wedge$  True is always equal to the truth value of p.
\end{property}

Although disjunction and conjunction appear to be two completey different operation, they are not. 
\begin{property}
$p${$\vee$}$q$ can be expressed as {$\neg$}({$\neg$}$p${$\wedge$}{$\neg$}$q$). 
$p${$\wedge$}$q$ can be expressed as {$\neg$}({$\neg$}$p${$\vee$}{$\neg$}$q$).

\end{property}



\begin{definition}[Inclusive Disjunction, Or operator] The inclusive disjunction of two propositions, denoted by p $\vee$ q, is read "p or q", is True when p is True or q is True, or both are True
\end{definition} 
The adjective \textit{disjunct} means \textit{disjoined}, {distinct from one another} \textit{combined}.
p: 4 is a number
q: A lion is a mammal 
r: January is a week


\begin{example}
p $\wedge$ q is True, as both p and q are True.
p $\wedge$ r is False, because not both p and r are True.
\end{example}

\begin{property}[Conjuction is commutative] The truth value of p $\vee$  q is equal to the truth value of q $\vee$ p.
\end{property}
In a compound conjuction p $\wedge$ q $\wedge$ r, the order of propositions has no effect on the truth value of overall truth value. The following propositions all have the same truth value:
p $\vee$ q $\vee$ r,
p $\vee$ r $\vee$ q,
q $\vee$ p $\vee$ r,
q $\vee$ r $\vee$ p,
r $\vee$ q $\vee$ p,
r $\vee$ p $\vee$ q
\begin{property}[Idempotency of p $\vee$  p] If p is True, p $\vee$ p is True. If p is False, then p $\vee$ p is False
\end{property}
Note that Disjunction itself is not idempotent: 
Not in all cases is  p $\vee$ q equal to p, e.g. when p is True and q is False: p $\vee$ q is False, but p is True
Not in all cases is  p $\vee$ q equal to q. 
But in all cases   p $\vee$ p is equal to p 

These two properties of disjunction can be used to simplify compound proposition into simpler ones or even a primitive one. This is in fact the core of logical reasoning. 
If somewhere in logic you end up with a proposition p $\wedge$ q $\wedge$ q $\wedge$ p, then by using these properties you can simplify this.

p $\vee$ q $\vee$ q $\vee$ p
equals to (Idempotency of q $\vee$ q) 
p $\vee$ q $\vee$ p
equals to  (Conjuction is commutative)
p $\vee$ p $\vee$ q
equals to (Idempotency of p $\vee$ p)
p $\vee$ q

The logical reduction from  p $\vee$ q $\vee$ q $\vee$ p to p $\vee$ q is valid in all cases, no matter what the truth value p or q.				


Disjunction is so widely used that is a built-in in all popular programming languages.\\
\begin{tabular}{ |c|c| }
\hline
Language & or \\
\hline
 C      &  \textbar\textbar  \\
 Java   & \textbar\textbar  \\
 Python & or   \\
 \hline
\end{tabular}

Conjunction is so widely used that is a built-in in all popular programming languages.\\

\begin{tabular}{ |c|c| }
\hline
Language & and \\
\hline
 C      &  \&\& \\
 Java   & \&\& \\
 Python & and    \\
 \hline
\end{tabular}\\ \\
These languages also have an operator \& for bitwise conjuction, we'll come to that in xxxx
% reference 
\begin{property}[p $\wedge$  True]  If any proposition in a disjunctive proposition is True, then the whole proposition is False. In other words: p $\vee$  True is always True.
\end{property}

This property is often used in computer languages to optimize code.

$$a = 5$$
$$b = 3$$
$$IF  a < 10 or b < 2 THEN do-something$$
		
If either condition $a< 10$ or $ b < 2$ are then the do-something part will be executed. 

Typically a programming language will evaluate the IF part from left to right. 
In this case $a < 10$ will be evaluated first. Since this is True, the remainder of the evaluation may be skipped and do-something will be executed.

\begin{tabular}{ |c|c|c|c| }
\hline
Language & not & and & or   \\
\hline
 C      & !   & \&\&  & \textbar\textbar  \\
 Java   &  !  &  \&\& &  \textbar\textbar \\
 Python & not & and   & or \\
 ksh    &     &   &  \\
 \hline
\end{tabular}

Xor is important for encryption and image processing

\section{Exercises}
\index{Exercise}
\begin{exercise}
Double negative: "We don't need no education"
On average human a human being has 10 fingers.

which is better not p or not q or not r     versus  not (p and q and r)
\end{exercise}
\begin{exercise}
(Dis)prove using truth tables:
\begin{itemize}
\item ($p${$\vee$}$q$){$\vee$}$r${$\leftrightarrow$}$p${$\vee$}($q${$\vee$}$r$)
\item ($p${$\rightarrow$}$q$){$\rightarrow$}$r${$\leftrightarrow$}$p${$\rightarrow$}($q${$\rightarrow$}$r$)
\end{itemize}
What is boys will be boys (Tautology)
Explain :s a tautology a paradox? 
No. A paradox doesn't just assert something incorrect (e.g. "0 not equal to 0") - it asserts something which cannot be consistently assigned a truth value. Just implying the negation of a tautology doesn't mean that a statement is paradoxical: e.g. "p and ¬p" is not a paradox, it's just a false statement


\end{exercise}
\chapter{}
\chapter{definitions}
\printindex



\end{document}

